\documentclass[12pt]{article}
\usepackage[russian]{babel}
\usepackage{indentfirst}
\usepackage{mathtools}
\usepackage{enumitem}
\usepackage[left=2cm, right=2cm, top=2cm, bottom=2cm, bindingoffset=0cm]
{geometry}

\begin{document}

\textbf{Белоброцкий Денис 4 курс 5 группа}
\\
\begin{center}
	{\Large Лабораторная работа №2}
\end{center} 
\begin{center}
	{\large \textbf{Моделирование дискретных СВ}}
\end{center} 
\begin{center}
	Вариант 2
\end{center}

	\section*{Постановка задачи}
	\begin{enumerate}
	\item
	Осуществить моделирование $ n = 1000 $ реализаций СВ для распределения Бернулли $ \overline{Bi}\left(1, 0.5\right) $ и отрицательного биномиального распределения $ \overline{Bi}\left(5, 0.25\right) $. 
	\item	
	Найти несмещённые оценки математического ожидания и дисперсии.
	\item	
	Проверить точность моделирования обоих распределений с помощью $ \chi^2 $-критерия Пирсона с уровнем значимости $ \varepsilon = 0.05 $. А также проверить, что вероятность ошибки $ I $ рода стремиться к $ 0.05 $.
	\end{enumerate}
	\section*{Теория}
	\par \textbf{Моделирование распределений последовательным методом обратных функций} 
	\par Распределение Бернулли и обратное биномиальное распределение можно задать в виде таблицы: \par
	\begin{tabular}{c c c c c}
		\hline
		0 & 1 & ... & $ k $ & ... \\
		$ p_0 $ & $ p_1 $ & ... & $ p_k $ & ... \\
		\hline
	\end{tabular}
	\\ где $ p_k = C^{k}_{n}p^k(1-p)^{n-k} $ в случае распределения Бернулли и $ p_k = C^{k}_{k+r-1}p^k(1-p)^r $ в случае отрицательного биномиального распределения.
	\par Для того, чтобы не хранить таблицу значений, а находить необходимое значение последовательно нам необходимо определить множитель на который изменяется $ p_k $ по отношению к $ p_{k-1} $, то есть $ \frac{p_k}{p_{k-1}} $. Найдём этот множитель для интересующих нас распределений:
	\par $ \frac{p_k}{p_{k-1}} = \frac{n-k+1}{k} \frac{p}{1-p} $, а $ p_0 = (1-p)^n $ в случае отрицательного биномиального распределения.
	\par $ \frac{p_k}{p_{k-1}} = \frac{k+r-1}{k}p $, а $ p_0 = (1-p)^r $ в случае отрицательного биномиального распределения.
	\par Алгоритм: \\
	\begin{enumerate}
		\item Инициализация и генерация $ \alpha $ с помощью БСВ: $ sum=probability=p_0, k=0, mul=\frac{p_k}{p_{k-1}} $
		\item Пересчет вероятностей и поиск "окна": $ while(\alpha > sum)\: do\; k=k+1, probability=probability*mul, sum=sum+probability $
		\item Завершение, результатом будет являться $ k $
	\end{enumerate}		
	
	\par \textbf{Критерий Пирсона}
	\par Данный критерий широко используется в задачах статистического анализа данных для проверки соответствия экспериментальных данных заданному модельному непрерывному или дискретному закону распределения, определяемому функцией распределения $ F_{0}(x)=F_{0}(x, \theta_{0}) $. 
	\par Гипотетические вероятности попадания значений $ \xi $ в ячейки гистограммы при истинной гипотезе $ H_{0} $ и полностью заданной функции $ F_{0}(x) $ равны:
$$
p_k = P \lbrace \xi \in [x_{k - 1}, x_k) \rbrace = F_0 (x_k) - F_0 (x_{k - 1}),
$$
где $ \lbrace x_l \rbrace (l = \overline{0, K}) $ - границы ячеек гистограммы. 
	\par Статистика критерия проверки гипотез имеет вид:
	\begin{equation}
		\chi^2 = \sum_{k = 1}^K \frac{(\nu_k - np_k)^2}{np_k} \geq 0 
	\end{equation}
	\par Данная статистика характеризует взвешенную сумму квадратов отклонений частот $ \lbrace \nu_k \rbrace $ от гипотетических значений $ \lbrace np_k \rbrace $. Чем больше $ \chi^2 $ , тем “сильнее” выборка $ X $ не согласуется с $ H_0 $. 
	\par Статистика $ (1) $ предполагает, что гипотеза $ H_0 $ верна, $ \chi^2 $ - распределение с $ K - 1 $ степенями свободы.
	\par Принятие гипотезы происходит следующим образом:
$$ 
	\begin{cases}
			H_{0},\; \chi^2 < \Delta, \\
			H_{1},\; \chi^2 \geq \Delta,
	\end{cases} 
$$
	\\где $ \Delta = 1 - F(\chi^2) $, $ F(\chi^2) $ - значение ф-ции распределения статистики $ (1) $ со степенью свободы $ K - 1 $. Значение это вычисляется с помощью какой-то функции, либо берется табличным. 

\end{document}